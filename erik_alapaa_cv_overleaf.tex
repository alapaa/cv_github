%-------------------------
% Rezume, a latex resume template for developers
% Author : Nanu Panchamurthy
% Based off of: https://github.com/sb2nov/resume
% License : MIT

% Hope this resume template helps you land an awesome job. If you found this
% helpful, please consider starring the github repo here, .

%-------------------------



%------------PACKAGES----------------
\documentclass[a4paper,11pt]{article}

\usepackage{verbatim} % reimplements the "verbatim" and "verbatim*"
                      % environments

\usepackage{titlesec} % provides an interface to sectioning commands
                      % i.e. custom elements

\usepackage{color} % provides both foreground and background color management

\usepackage{enumitem} % provides control over enumerate, itemize and
                      % description

\usepackage{fancyhdr} % provides extensive facilities for constructing headers,
                      % footers and also controlling their use

\usepackage{tabularx} % defines an environment tabularx, extension of "tabular"
                      % with an extra designator x, paragraph like column whose
                      % width automatically expands to fill the width of the
                      % environment

\usepackage{latexsym} % provides mathematical symbols

\usepackage{marvosym} % provides martin vogel's symbol font which contains
                      % various symbols

\usepackage[empty]{fullpage} % sets margins to one inch and removes headers,
                             % footers etc..

\usepackage[hidelinks]{hyperref} % removes color and shadow of hyperlinks

\usepackage[normalem]{ulem} % provides "\ul" (uline) command which will break
                            % at line breaks

\usepackage[english]{babel} % provides culturally determined typographical
                            % rules for wide range of languages
%-----------------------------------------

\input glyphtounicode % converts glyph names to unicode
\pdfgentounicode=1 % ensures pdfs generated are ats readable

%----------FONT OPTIONS-------------------
\usepackage[default]{sourcesanspro} % uses the font source sans pro
\urlstyle{same} % changes url font from default urlfont to font being used by
                % the document
%-----------------------------------------


%----------MARGIN OPTIONS-----------------
\pagestyle{fancy} % set page style to one configured by fancyhdr
\fancyhf{} % clear all header and footer fields

\renewcommand{\headrulewidth}{0in} % sets thickness of linerule under header to
                                   % zero
\renewcommand{\footrulewidth}{0in} % sets thickness of linerule over footer to
                                   % zero

\setlength{\tabcolsep}{0in} % sets thickness of column separator in tables to
                            % zero

% origin of the document is one inch from the top and from and the left
% oddsidemargin and evensidemargin both refer to the left margin
% right margin is indirectly set using oddsidemargin
\addtolength{\oddsidemargin}{-0.5in}
\addtolength{\topmargin}{-0.5in}

\addtolength{\textwidth}{1.0in} % sets width of text area in the page to one
                                % inch
\addtolength{\textheight}{1.0in} % sets height of text area in the page to one
                                 % inch

\raggedbottom{} % makes all pages the height of current page, no extra vertical
                % space added
\raggedright{} % makes all pages the width of current page, no extra horizontal
               % space added
%------------------------------------------


%--------SECTIONING COMMANDS---------
% \titleformat{<command>}
%   [<shape>]{<format>}{<label>}{<sep>}
%     {<before-code>}[<after-code>]

% command is the sectioning command to be redefined shape is the style of the
% font; scshape stands for small caps style format is the format to be applied
% to whole title- label and text; absent here label defines the label sep is
% the horizontal separation between label and title body before-code is the
% code to be executed before after-code is the code to be executed after

\titleformat{\section}
  {\scshape\large}{}
    {0em}{\color{blue}}[\color{black}\titlerule\vspace{0pt}]
%-------------------------------------


%--------REDEFINITIONS----------------
% redefines the style of the bullet point
\renewcommand\labelitemii{$\vcenter{\hbox{\tiny$\bullet$}}$}

% redefines the underline depth to 2pt
\renewcommand{\ULdepth}{2pt}
%-------------------------------------


%--------CUSTOM COMMANDS--------------

%\vspace{} defines a vertical space of given size, modifying this in custom
%commands can help stretch or shrink resume to remove or add content

% resumeItem renders a bullet point
\newcommand{\resumeItem}[1]{
  \item\small{#1}
}

% commands to start and end itemization of resumeItem, rightmargin set to
% 0.11in to avoid the overflow of resumetItem beyond whatever resumeItemHeading
% is being used

\newcommand{\resumeItemListStart}{\begin{itemize}[rightmargin=0.11in]}
\newcommand{\resumeItemListEnd}{\end{itemize}}

% resumeSectionType renders a bolded type to be used under a section, used as
% skill type here, middle element is used to keep ":"s in the same vertical
% line

\newcommand{\resumeSectionType}[3]{
  \item\begin{tabular*}{0.96\textwidth}[t]{
    p{0.15\linewidth}p{0.02\linewidth}p{0.81\linewidth}
  }
    \textbf{#1} & #2 & #3
  \end{tabular*}\vspace{-2pt}
}

% resumeTrioHeading renders three elements in three columns with second element
% being italicized and first element bolded, can be used for projects with
% three elements

\newcommand{\resumeTrioHeading}[3]{
  \item\small{
    \begin{tabular*}{0.96\textwidth}[t]{
      l@{\extracolsep{\fill}}c@{\extracolsep{\fill}}r
    }
      \textbf{#1} & \textit{#2} & #3
    \end{tabular*}
  }
}

% resumeQuadHeading renders four elements in a two columns with the second row
% being italicized and first element of first row bolded, can be used for
% experience and projects with four elements

\newcommand{\resumeQuadHeading}[4]{
  \item
  \begin{tabular*}{0.96\textwidth}[t]{l@{\extracolsep{\fill}}r}
    \textbf{#1} & #2 \\
    \textit{\small#3} & \textit{\small #4} \\
  \end{tabular*}
}

% resumeQuadHeadingChild renders the second row of resumeQuadHeading, can be
% used for experience if different roles in the same company need to added

\newcommand{\resumeQuadHeadingChild}[2]{
  \item
  \begin{tabular*}{0.96\textwidth}[t]{l@{\extracolsep{\fill}}r}
    \textbf{\small#1} & {\small#2} \\
  \end{tabular*}
}

% commands to start and end itemization of resumeQuadHeading, lefmargin for
% left indent of 0.15in for resumeItems

\newcommand{\resumeHeadingListStart}{
  \begin{itemize}[leftmargin=0.15in, label={}]
}
\newcommand{\resumeHeadingListEnd}{\end{itemize}}
%-------------------------------------------


%__________________RESUME____________________
% You can rearrange sections in any order you may prefer
\begin{document}

%-----------CONTACT DETAILS------------------
% Make sure all the details are correct, you can add more links in the first
% row of second column if needed


\begin{tabular*}{\textwidth}{l@{\extracolsep{\fill}}r}
  \textbf{\Huge Erik Alapää \vspace{2pt}} & % row = 1, col = 1
  Location: Zurich, Switzerland \\ % row = 1, col = 2

  \href{https://linkedin.com/in/alapaa/}{\uline{linkedin.com/in/alapaa/}}
  $|$ % row = 2, col = 1

  Email: \href{mailto:erik.alapaa@gmail.com}{\uline{erik.alapaa@gmail.com}}
  $|$ % row = 2, col = 2

  Mobile: +41 77 265 14 66 \\ % row = 2, col = 2
\end{tabular*}
%--------------------------------------------


%-----------SUMMARY--------------------------

% Keep this short, simple and straigth to point

\section{Hands-on, research-trained tech lead, software developer and
  near-metal C++ expert at Google}
\small{ Software developer in Borg(*) SRE at Google. Specialist in Unix and hard
  real-time C/C++. Research degree (Licentiate) in Applied
  Mathematics from Chalmers, Gothenburg. M.Sc. in Engineering Physics from
  Luleå University of Technology. +20 years of industry experience, mostly on
  \textbf{near-metal programming in C and C++}. Worked on everything from
  \textbf{192k RAM} embedded controllers up to Google's world-scale
  infrastructure.

  (*) \emph{Borg is the cluster operating system that controls all Google
  datacenters and runs almost all infrastructure at Google.}  }
%--------------------------------------------


%--------------SKILLS------------------------
% Add or remove resumeSectionTypes according to your needs
\section{Roles I am looking for: Pure SWE, research, architect, tech lead.}

\section{Technical Skills}
\resumeHeadingListStart{}

    \resumeSectionType{Languages}{:}{C++, C, Python}
    \resumeSectionType{Databases}{:}{Google Cloud BigQuery}
    \resumeSectionType{Dev Tools}{:}{Clang, GCC, GDB, Git, Valgrind (Memcheck,
      Cachegrind, Callgrind, Helgrind), Clang
      Thread Safety Analysis, Compiler Explorer (godbolt.org)}
    \resumeSectionType{Frameworks}{:}{Boost Graph Library, Pthreads, Google
      Protobuf (open source), Google Abseil C++ Common Libraries (open source)}

\resumeHeadingListEnd{}
%--------------------------------------------


%-----------EXPERIENCE-----------------------

% Distill all your talking points to small bullet points which follow the
% pattern "challenge-action-result" for maximum efficiency. Try to quantify
% (use numbers) your points whenver possible, highlist words of importance

\section{
  Career highlights, examples of my expertise
  (full history at linkedin.com/in/alapaa/)}
\resumeHeadingListStart{}

   \resumeQuadHeading{SRE, Borg Cluster Operating System}{Feb 2019 -- Present}
   {Google}{Zurich, Switzerland}
   \resumeItemListStart{}

     \resumeItem{Single-handedly developed automation for the machine
       management part of Borg turnup, removing a very time-consuming process
       for my team's oncallers. Turned up 50 Google Edge clusters in Q1 2021,
       dramatically improving the sharding, redundancy and reliability of
       Google's Edge cluster network.}

     \resumeItem{As the \textbf{Edge cluster turnup lead} of a small team of
       SREs developed and deployed improved automation that \textbf{entirely
         removed cluster controller machine management toil and costs} (by
       moving to a model where cluster controllers do not need dedicated
       machines).}

     \resumeItem{Designed and implemented a solution for turndown cleanup in
       the Aristotle (C++ based) machine management system. This saved Google
       very large amounts of money by not blocking subsequent large cluster
       turnups, and also eliminated hours of toil every week for me and other
       oncallers in the SRE team. I also designed safety checks preventing the
       cleanup to be run on production cells, which could have caused large
       outages.}

     \resumeItem{Modified handling of open-source Protocol Buffer
       \href{https://github.com/protocolbuffers/protobuf/commit/562fc946c748d0ddb451ad1ce48570685f23a4f1}
       {\uline{extension code}} (C++), saving a couple of bytes per message. This
       \textbf{saved 3 TB of RAM across the Google datacenter fleet}.}

     \resumeItem{\textbf{Taught all 3 Google C++ courses} to hundreds of Google
       engineers during 2021-2023.}

    \resumeItemListEnd{}

\resumeQuadHeading{TSE, Technical Solutions Engineer}{Aug 2017 -- Jan 2019}
  {Google}{Zurich, Switzerland}
  \resumeItemListStart{}

      \resumeItem{3rd line expert support of Google Cloud Big Data products
        (e.g. BigQuery, Dataflow, Pub/Sub, Tensorflow)}.

      \resumeItem{Specializing in the architecture of
        \href{https://panoply.io/data-warehouse-guide/bigquery-architecture}{\uline{BigQuery}},
        an extremely fast, mostly-read SQL database, handling
        \textbf{terabytes} of queries in seconds.}

      \resumeItem{Worked with premium customers, developers and SRE to
        \textbf{debug the C++ shuffler part of the database engine}, solving
        slowdowns and failures of large queries and speeding up query execution
        by a factor of \textbf{10-20 times}.}

      \resumeItemListEnd{}

\resumeQuadHeading{C and C++ SW developer, tech lead}{Jan 2014 -- Jun 2017}
  {Netrounds}{Luleå, Sweden}    %
  \resumeItemListStart{}

      \resumeItem{Occasional \textbf{Linux kernel patches} in CPE:s (home
        routers/WiFi access points).} One prominent example was when I added
      support to the Broadcom kernel and user-space API for tapping the built
      in layer 2 hardware (the switch), making packets visible to the host
      kernel. This enabled our measurement software to access the packets in
      user-space.

      \resumeItem{Research and hands-on development in C++ of hardware
        timestamping of network packets, \textbf{improving accuracy and
          precision of measurements by a factor of 1000}. I did all the
        research, wrote the requirements specification and wrote the
        time-critical core measurement functions and also lead a small team of
        developers in this effort.}

      \resumeItem{Extensive study of the Linux kernel networking code, for
        example to \textbf{understand the path MTU detection logic} and use it
        fully in our custom IP measurement software.}

      \resumeItem{Developed a solution using SR-IOV for \textbf{near bare metal
          latency RTT measurements} by mapping the hardware queues of a network
        interface card directly to a virtual machine, bypassing the host
        kernel.}

    \resumeItemListEnd{}

\resumeQuadHeading{C++ SW developer, consultant}{2012-2013}
  {Ericsson}{Stockholm, Sweden}
    \resumeItemListStart{}

      \resumeItem{\textbf{Advanced debugging of multi-threaded software} in the
        mobile base station radio software. (Pthreads and OSE RTOS
        processes)}. One example was finding an initialization bug caused by an
      older C++ compiler not emitting locks around static initialization at the
      startup of the system.

      \resumeItem{C++ development in an extremely constrained \textbf{hard
          real-time} environment, \textbf{64K data RAM and 128k instruction
          RAM}, the system for linearization in the mobile radio base station
        transmitter}. Using my research background to interact with
      mathematicians and signal processing experts to realize the
      implementation. Working hands-on in the lab verifying the implementation
      directly on the target RISC CPU.

    \resumeItemListEnd{}

\resumeQuadHeading{Tech lead, SW developer (pure C), consultant}{2010-2011}
  {Combitech/Ericsson}{Stockholm, Sweden}
  \resumeItemListStart{}

      \resumeItem{Shared tech lead of a project to develop a distributed,
        scalable, telecom-grade load balancer using the built-in Linux kernel
        load balancer for \textbf{+50 million connections},} including
      termination of hardware-encrypted IPSec connections.

      \resumeItem{I wrote the requirements specification for the \textbf{Fault
          Tolerance and High Availability} design of the load balancer system.}

      \resumeItem{\textbf{Developed a coroutines solution in C} for the control
        plane of the load balancer to avoid spawning new threads when operators
        logged in to configure the system using a Cisco-style router CLI.}

      \resumeItem{Worked with Ericsson legal experts on \textbf{open-source
          licensing} to deploy the coroutines solution.}

      \resumeItem{The load balancer project contributed fixes to the then-new
        \textbf{Linux kernel container/network namespace code} (for example
        fixing issues in the VETH interface logic), and most prominently, the
        HMARK patch in the kernel firewall (Netfilter) code that is
        foundational for load balancing in Linux.}  \resumeItemListEnd{}

\resumeHeadingListEnd{}
%---------------------------------------------


%-----------EDUCATION-------------------------
% Mention your CGPA, if its good, in the first row of second column

\section{Education}
  \resumeHeadingListStart{} \resumeQuadHeading{Chalmers University of
    Technology}{Gothenburg, Sweden}{Licentiate \textbf{(research degree) in
      Applied Mathematics}. Specialization in wavelets and Fourier
    analysis.}{Aug 1999 -- Nov 2004}

    \href{https://research.chalmers.se/en/publication/94396}{\uline{Thesis}} on
    partial differential equations and error-correcting codes for efficient
    transmission of multimedia over 4G mobile radio (Ericsson Resarch),
    implemented in C++.

    \resumeQuadHeading{Luleå Technical University}{Luleå, Sweden} {Master of
      Science in Engineering Physics, specializing in Applied Mathematics}{Aug
      1993 -- Aug 1999}

    \href{https://ltu.diva-portal.org/smash/get/diva2:1032007/FULLTEXT01.pdf}
         {\uline{Thesis}} on surface reconstruction using laser-scanned point
         cloud samples, implemented in C++.  \resumeHeadingListEnd{}
%---------------------------------------------


%-----------PROJECTS--------------------------

% Use resumeQuadHeading if four elements are feasible (ex: demo video link),
% else use resumeTrioHeading. Keep the bullet points simple and concise and try
% to cover wide variety of skills you have used to build these projects

%--------------------------------------------


%----------------OTHERS----------------------
% You can add your acheivements, accolades, certifications etc. here.

\section{Certifications}
\resumeItemListStart{}


  \resumeItem{\textbf{C++ master level, Brainbench}.}

  \resumeItem{Achieved Google C++ readability, allowing me to submit
    changelists without separate readability review.}

  \resumeItemListEnd{}

\section{Additional information}
\resumeItemListStart{}

    \resumeItem{Nationality: Swedish, lived in Switzerland the last 6 years
      with my wife and 9-year old son. Grew up above the Arctic Circle.}

    \resumeItem{Got my first computer,
      \href{https://en.wikipedia.org/wiki/ABC_80}{\uline{Electrolux ABC 80}},
      in 1982. \textbf{Started writing M68K assembler on my Amiga 500
        1988}. First full-time position as SW developer at ABB Signal,
      Stockholm 1992, developing fail-safe hard real-time systems for
      trains. Started learning and using C++ professionally 1993.}

    \resumeItem{\textbf{Six languages}: Swedish (native), English (excellent),
      Russian (fluent spoken Russian), German (very good understanding, only
      rudimentary spoken German), Finnish (very basic), French (very basic)}

    \resumeItem{Interests: Political economy, economic history, technical
      history, energy technology including solar, nuclear and hydrogen
      storage. Avid e-biker.}  \resumeItemListEnd{}

%--------------------------------------------
\end{document}
